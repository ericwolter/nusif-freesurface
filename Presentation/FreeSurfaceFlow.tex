\documentclass{beamer}
\usepackage[utf8]{inputenc}
\usepackage[T1]{fontenc}
\usepackage{lmodern}
\usepackage{graphicx}
\usepackage{amsmath, amssymb, graphicx}
%\usepackage{beamerthemeshadow}
%\beamersetuncovermixins{\opaqueness<1>{25}}{\opaqueness<2->{15}}
\usetheme{Ilmenau}
\begin{document}
\title{Free Surface Flows} 
\date{03.02.2014}
\author{M. Farahani, E. Wolter, A. Hahn}
\frame{\titlepage}
%\frame{\frametitle{Structure}\tableofcontents} 


\section[Problem]{Problem description} 
%\subsection{}
\frame{ %\frametitle{ } 
Finde eine st\"uckweise zwei mal stetig differenzierbare Bahnkurve der Hinterachse $\Phi \in C^1([0,t^*],\mathbb{R}^2)$, sodass\\
\begin{enumerate}[(I)]
\item das Auto zu jedem Zeitpunkt $t$ in einem Gebiet $G$ ist
\item die Randbedingungen f\"ur $\Phi(0), [\Phi(t^*)]_2$ erf\"ullt sind und
  $\frac{\Phi'(0)}{\|\Phi'(0)\|} = \frac{\Phi'(t^*)}{\|\Phi'(t^*)\|} =\begin{pmatrix}-1 \\ 0 \end{pmatrix}$
\item f\"ur $-\frac{\Phi^{'}(t)}{\| \Phi^{'}(t) \|_2} = \begin{pmatrix} \cos \beta(t)\\ \sin \beta(t) \end{pmatrix}$\\
$\beta(t) \in \left[0, \frac{\pi}{2}\right[ \mbox{f\"ur alle  } t \in \left[0,t^*\right]$ erf\"ullt ist
\item (Wendekreisbeschr\"ankung erf\"ullt)
  
\end{enumerate}
}


\section[Creating new functions ]{Implementation} 
\frame{\frametitle{New functions}
%Es ist $x(t) = \Phi(t) + \begin{pmatrix} (l-h)\cos \beta(t) - \frac{b}{2} (-\sin \beta(t)) \\
%-h\sin \beta(t)-\frac{b}{2} \cos \beta(t)\end{pmatrix}$.
\begin{figure}[ht]
	\centering
	 \fbox{
  %\includegraphics[width=0.8\textwidth]{bilder/A_in_G_standart.pdf}
	%\caption{um 30 Grad gedreht}
	\label{fig1}
	}
\end{figure}
}

\frame{\frametitle{Changing old functions}
%Es ist $x(t) = \Phi(t) + \begin{pmatrix} (l-h)\cos \beta(t) - \frac{b}{2} (-\sin \beta(t)) \\
%-h\sin \beta(t)-\frac{b}{2} \cos \beta(t)\end{pmatrix}$.
\begin{figure}[ht]
	\centering
	 \fbox{
  %\includegraphics[width=0.8\textwidth]{bilder/A_in_G_1.pdf}
	%\caption{um 30 Grad gedreht}
	\label{fig1}
	}
\end{figure}
}

\frame{\frametitle{Veranschaulichung Bedingung $(ii)$}
%Es ist $x(t) = \Phi(t) + \begin{pmatrix} (l-h)\cos \beta(t) - \frac{b}{2} (-\sin \beta(t)) \\
%-h\sin \beta(t)-\frac{b}{2} \cos \beta(t)\end{pmatrix}$.
\begin{figure}[ht]
	\centering
	 \fbox{
  %\includegraphics[width=0.8\textwidth]{bilder/A_in_G_2.pdf}
	%\caption{um 30 Grad gedreht}
	\label{fig1}
	}
\end{figure}

}

\section{Results}
\subsection{The Breaking Dam}
\frame{\frametitle{Breaking dam with outflow at the east wall}
\begin{equation*} \label{Ansatz Geraden Kreise}
\Phi(t)= 
\left\{ 
\begin{aligned} 
\Phi(0) - 
\begin{pmatrix}
1\\0\\
\end{pmatrix}
	t && \text{ für } t\in [0,t_0] \\
M^1 + r_1 
\begin{pmatrix}
-\sin\left(\frac{t-t_0}{r_1}\right)\\
\cos\left(\frac{t-t_0}{r_1}\right)
\end{pmatrix}
	&& \text{ für } t\in [t_0,t_1] \\
\Phi(t_1) +  
\begin{pmatrix}
-\cos\left(\frac{t_1-t_0}{r_1}\right)\\
-\sin\left(\frac{t_1-t_0}{r_1}\right)
\end{pmatrix}
(t-t_1)
	&& \text{ für } t\in [t_1,t_2] \\
M^2 + r_2 
\begin{pmatrix}
\sin\left(\frac{t^*-t}{r_2}\right)\\
-\cos\left(\frac{t^*-t}{r_2}\right)
\end{pmatrix}
	&& \text{ für } t\in [t_2, t^*] \\
\end{aligned} 
\right.
\end{equation*}
}

\frame{\frametitle{Breaking dam with free-slip at the east wall}
%Es ist $x(t) = \Phi(t) + \begin{pmatrix} (l-h)\cos \beta(t) - \frac{b}{2} (-\sin \beta(t)) \\
%-h\sin \beta(t)-\frac{b}{2} \cos \beta(t)\end{pmatrix}$.
\begin{figure}[ht]
	\centering
	 \fbox{
  %\includegraphics[width=0.8\textwidth]{bilder/Phi_ansatz.pdf}
	%\caption{um 30 Grad gedreht}
	\label{fig1}
	}
\end{figure}

}

\subsection{The Splash of a Liquid Drop}
\frame{\frametitle{Falling drop}
\begin{itemize}
\item Umschreibung der Bedingungen in die Variablen $t_0,t_1,r_1$ und $t_2$.
\item L\"osung des entstehenden Optimierungsproblems
\end{itemize}
}

\end{document}
